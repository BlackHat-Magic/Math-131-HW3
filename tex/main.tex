\documentclass[12pt,twoside]{amsart}

% formatting stuff
\addtolength{\hoffset}{-2.75cm}
\addtolength{\textwidth}{5.5cm}
\addtolength{\voffset}{-2.5cm}
\addtolength{\textheight}{3.5cm}
\setlength{\headheight}{17pt}

% \usepackage{amsthm}
% \usepackage{amsmath}
\usepackage{amssymb}
\usepackage[colorlinks = true, linkcolor = black, citecolor = black, final]{hyperref}
\usepackage{enumerate}

\usepackage{graphicx}
% \usepackage{multicol}

\usepackage{paracol}
\setlength{\parskip}{0pt}
\setlength{\parindent}{0pt}

% \usepackage{ marvosym }
% \usepackage{wasysym}
% \usepackage{tikz}
% \usetikzlibrary{patterns}

\usepackage{fancyhdr}

\usepackage{listings}
\usepackage{xcolor}
\definecolor{codegreen}{rgb}{0,0.6,0}
\definecolor{codegray}{rgb}{0.5,0.5,0.5}
\definecolor{codepurple}{rgb}{0.58,0,0.82}
\definecolor{backcolour}{rgb}{0.95,0.95,0.92}

\lstdefinestyle{mystyle}{
    backgroundcolor=\color{backcolour},   
    commentstyle=\color{codegreen},
    keywordstyle=\color{magenta},
    % numberstyle=\tiny\color{codegray},
    stringstyle=\color{codepurple},
    basicstyle=\ttfamily\footnotesize,
    breakatwhitespace=false,         
    breaklines=true,                 
    captionpos=b,                    
    keepspaces=true,                 
    numbers=none,                    
    numbersep=5pt,                  
    showspaces=false,                
    showstringspaces=false,
    showtabs=false,                  
    tabsize=2
}
\lstset{style=mystyle}

\newcommand{\ds}{\displaystyle}

\setlength{\parindent}{0in}

\pagestyle{empty}

% ----------------------------
% end preamble
% -----------------------------

\begin{document}

\thispagestyle{fancy}
\pagestyle{fancy}
\fancyhf{}
\fancyhead[LE,LO]{\scshape Math 131}
\fancyhead[CE,CO]{\scshape \large Assignment 3}
\fancyhead[RE,RO]{\scshape Luke Henderson}

For the source code for the coding problems, see the attached Jupyter notebook, Matlab live script. Alternatively, the entire git repository  is attached as a zip archive, and is available \href{https://github.com/blackHat-Magic/math-131-HW3}{on GitHub}. The comments in the code have been omitted here for brevity. They are present in the Jupyter notebook and Matlab live script.

\begin{enumerate}
    \item \textbf{Problem Statement:} Compute the interpolating polynomial $p_2(x)$ that interpolates $f = \sqrt{2}xcos(x)$ at points $x_0 = 0$, $x_1 = \frac{\pi}{4}$, and $x_2 = \frac{\pi}{2}$ in the interval $[0, \frac{\pi}{2}]$.
    \begin{align*}
    a(x_0)^2 + b(x_0) + c & = 0 \\
    a(x_1)^2 + b(x_1) + c & = \frac{\pi}{4} \\
    a(x_2)^2 + b(x_2) + c & = 0 \\
    \end{align*}

    We can solve this as a matrix-vector equation.

    \begin{align*}
    \left[\begin{array}{ccc}
        0 & 0 & 1 \\ 
        \frac{\pi^2}{16} & \frac{\pi}{4} & 1 \\ 
        \frac{\pi^2}{4} & \frac{\pi}{2} & 1
    \end{array}\right] \times \left[\begin{array}{c}
        a \\
        b \\
        c
    \end{array}\right] & = \left[\begin{array}{c}
        0 \\
        \frac{\pi}{4} \\
        0
    \end{array}\right] \\
    \left[\begin{array}{ccc|c}
        0 & 0 & 1 & 0 \\
        \frac{\pi^2}{16} & \frac{\pi}{4} & 1 & \frac{\pi}{4} \\
        \frac{\pi^2}{4} & \frac{\pi}{2} & 1 & 0
    \end{array}\right] & \text{swap(R1, R3)} \\
    \left[\begin{array}{ccc|c}
        \frac{\pi^2}{4} & \frac{\pi}{2} & 1 & 0 \\
        \frac{\pi^2}{16} & \frac{\pi}{4} & 1 & \frac{\pi}{4} \\
        0 & 0 & 1 & 0 \\
    \end{array}\right] & \text{swap(R1, R2)} \\
    \left[\begin{array}{ccc|c}
        \frac{\pi^2}{16} & \frac{\pi}{4} & 1 & \frac{\pi}{4} \\
        \frac{\pi^2}{4} & \frac{\pi}{2} & 1 & 0 \\
        0 & 0 & 1 & 0 \\
    \end{array}\right] & \text{R2} = \text{R2} - 4 \times \text{R1} \\
    \left[\begin{array}{ccc|c}
        \frac{\pi^2}{16} & \frac{\pi}{4} & 1 & \frac{\pi}{4} \\
        0 & -\frac{\pi}{2} & -3 & -\pi \\
        0 & 0 & 1 & 0 \\
    \end{array}\right]
    \end{align*}

    Now we can solve for $a$, $b$, and $c$. \\
    
    \begin{align*}
        1(c) & = 0 \\
        c & = 0 \\
    \end{align*}

    \begin{align*}
        -\frac{\pi}{2}b - 3c & = -\pi \\
        -\frac{\pi}{2}b & = -\pi \\
        b & = 2 \\
    \end{align*}

    \begin{align*}
        \frac{\pi^2}{16}a + \frac{\pi}{4}b + 1c & = \frac{\pi}{4} \\
        \frac{\pi^2}{16}a + \frac{\pi}{4}2 + 1(0) & = \frac{\pi}{4} \\
        \frac{\pi^2}{16}a + \frac{\pi}{2} & = \frac{\pi}{4} \\
        \frac{\pi^2}{16}a & = -\frac{\pi}{4} \\
        \pi^2a & = -4\pi \\
        a & = \frac{-4}{\pi}
    \end{align*}
    
    Now that we have the coefficients, we can put them together to find the Lagrange polynomial $p_2(x) = -\frac{4}{\pi}x^2 + 2x$
    \bigskip

    \item\textbf{Problem Statement:} U.S. populations for the years 1990, 2000, and 2010, is given in the table
    \begin{center}
    \begin{tabular}{|c||c|c|c|}
        \hline
        $x$ (year) & $1990$ & $2000$ & $2010$ \\
        \hline
        $y$ (population) & $248,709,873$ & $281,421,906$ & $308,745,538$ \\
        \hline
    \end{tabular}
    \end{center}
    Estimate the population in 2006 using
    \begin{enumerate}
        \item Linear Splines
        \begin{align*}
            y(2006) & \approx \frac{y(2010) - y(2000)}{2010 - 2000}(2006 - 2000) + 281,421,906 \\
            & \approx \frac{308,745,538 - 281,421,906}{10}6 + 281,421,906 \\
            & \approx \frac{27,323,632}{10}6 + 281,421,906 \\
            & \approx 2,732,363.2 \times 6 + 281,421,906 \\
            y(2006) & \approx 297,816,085
        \end{align*}
        \item $p_2(x)$ that interpolates the population of the U.S. at the data points $x_0 = 1990$, $x_1 = 2000$, and $x_2 = 2010$.
        \begin{align*}
            \left[\begin{array}{ccc|c}
                3,960,100 & 1990 & 1 & 248,709,873 \\
                4,000,000 & 2000 & 1 & 281,421,906\\
                4,040,100 & 2010 & 1 & 308,745,538
            \end{array}\right] & \begin{array}{c}\text{R2} = \text{R2} - 1.010075503\text{R1} \\ \text{R3} = \text{R3} - 1.020201510\text{R1}\end{array} \\
            \left[\begin{array}{ccc|c}
                3,960,100 & 1990 & 1 & 248,709,873 \\
                0 & -10.05025097 & -0.010075503 & 30,206,155.93 \\
                0 & -20.2010049 & -0.020201510 & 55,011,350.01
            \end{array}\right] & \text{R3} = \text{R3} - 2.010000045\text{R2} \\
            \left[\begin{array}{ccc|c}
                3,960,100 & 1990 & 1 & 248,709,873 \\
                0 & -10.05025097 & -0.010075503 & 30,206,155.93 \\
                0 & 0 & 0.00005025148340 & -5,703,024.769
            \end{array}\right]
        \end{align*}

        \begin{align*}
            0.00005025148340c & = -5703024.769 \\
            c & = -113489679968.33304
        \end{align*}

        \begin{align*}
            -10.05025097b - 0.010075503c & = 30206155.93 \\
            -10.05025097b + 1143465610.9899793 & = 30206155.93 \\
            -10.05025097b & = -1113259455.0599792 \\
            b & = 110769318.93373197
        \end{align*}

        \begin{align*}
            3960100a + 1990b + c & = 248709873 \\
            3960100a + 220430944678.12662 - 113489679968.33304 & = 248709873
            3960100a & = -106692554836.79358 \\
            a & = -26941.884002119536
        \end{align*}

        $p_2(x) = -26941.884x^2 + 110769318.93373x - 113489679968.33304$ \\
        $p_2(2006) = 298,462,693$
    \end{enumerate}
    \bigskip

    \item\textbf{Problem Statement:} Find $b$, $c$, and $d$ such that
    $$s(x) = \begin{cases}1 + 2x - x^3, & 0 \leq x \leq 1 \\ 2 + b(x - 1) + c(x - 1)^2 + d(x - 1)^3, & 1 < x \leq 2\end{cases}$$
    is a natural cubic spline. \\

    $$s'(x) = \begin{cases}2 - 3x^2, & 0 \leq x \leq 1 \\ b + 2c(x - 1) + 3d(x - 1)^2, & 1 < x \leq 2\end{cases}$$
    \begin{align*}
        \underset{x \rightarrow 1^+}\lim{s'(x)} = \underset{x \rightarrow 1^-}\lim{s'(x)} \implies 2 - 3(1)^2 & = b - 2c(1 - 1) + 3d(1 - 1)^2 \\
        2 - 3 & = b - 2c(0) + 3d(0)^2 \\
        -1 & = b \\
        b & = -1
    \end{align*}

    $$s''(x) = \begin{cases}-6x, & 0 \leq x \leq 1 \\ 2c + 6d(x - 1), & 1 < x \leq 2\end{cases}$$
    \begin{align*}
        \underset{x \rightarrow 1^+}\lim{s''(x)} = \underset{x \rightarrow 1^-}\lim{s''(x)} \implies -6(1) & = 2c + 6d(1 - 1) \\
        -6 & = 2c \\
        c & = -3
    \end{align*}
    \begin{align*}
        s''(0) = s''(2) = 0 \implies -6(0) = 2(-3) + 6d(2 - 1) & = 0 \\
        -6 + 6d(1) & = 0 \\
        6d = 6 \\
        d = 1
    \end{align*}
    \bigskip

    \item\textbf{Problem Statement:} The following function $s$ is a cubic spline that interpolates the following values. Find $a$, $b$, $c$, $d$, $y_1$, and $y_2$.
    \begin{align*}
    s(x) & = \begin{cases}
        \frac{5}{4}(x + 2)^3 - \frac{9}{2}(x + 2)^2 + 8, & -2 \leq x < 0 \\
        ax^3 + bx^2 + cx + d, & 0 \leq x < 1 \\
        (x - 1)^3 - 1, & 1 \leq x < 2 \\
        -\frac{5}{4}(x - 2)^3 + 3(x - 2)^2 + 3(x - 2), & 2 \leq x \leq 4
    \end{cases} \\
    s'(x) & = \begin{cases}
        \frac{15}{4}(x + 2)^2 - 9(x + 2), & -2 \leq x < 0 \\
        3ax^2 + 2bx + c, & 0 \leq x < 1 \\
        3(x - 1)^2, & 1 \leq x < 2 \\
        -\frac{15}{4}(x - 2)^2 + 6(x - 2), & 2 \leq x \leq 4
    \end{cases} \\
    s''(x) & = \begin{cases}
        \frac{15}{2}(x + 2) - 9, & -2 \leq x < 0 \\
        6ax + 2b, & 0 \leq x < 1 \\
        6(x - 1), & 1 \leq x < 2 \\
        -\frac{15}{2}(x - 2) + 6, & 2 \leq x \leq 4
    \end{cases} \\
    \end{align*}

    \begin{center}
    \begin{tabular}{|c||c|c|c|c|c|}
    \hline
    $x$ & $-2$ & $0$ & $1$ & $2$ & $4$ \\
    \hline
    $y$ & $8$ & $y_1$ & $y_2$ & $0$ & $8$ \\
    \hline
    \end{tabular}
    \end{center}

    \begin{align*}
    \underset{x \rightarrow 0^+}\lim{s''(x)} = \underset{x \rightarrow 0^-}\lim{s''(x)} & \implies \frac{15}{2}(0 + 2) - 9 = 6a(0) + 2b \\
    15 - 9 & = 2b \\
    6 & = 2b \\
    b & = 3
    \end{align*}
    \begin{align*}
    \underset{x \rightarrow 1^+}\lim{s''(x)} = \underset{x \rightarrow 1^-}\lim{s''(x)} & \implies 6(1 - 1) = 6a(1) + 2b\\
    0 & = 6a + 2(3) \\
    -6 & = 6a \\
    a & = -1
    \end{align*}
    \begin{align*}
    \underset{x \rightarrow 1^+}\lim{s'(x)} = \underset{x \rightarrow 1^-}\lim{s'(x)} & \implies 3a(1)^2 + 2b(1) + c = 3(1 - 1)^2 \\
    3(-1) + 2(3) + c & = 3(0) \\
    6 - 3 + c & = 0 \\
    3 + c & = 0 \\
    c & = -3
    \end{align*}
    \begin{align*}
        \underset{x \rightarrow 1^+}\lim{s(x)} = \underset{x \rightarrow 1^-}\lim{s(x)} & \implies a(1)^3 + b(1)^2 + c(1) + d = (1 - 1)^3 - 1 \\
        -1 + 3 - 3 + d & = -1 \\
        d & = 0
    \end{align*}
    \begin{align*}
    y_1 = s(0) & = ax^3 + bx^2 + cx + d \\
    & = -1(0)^3 + 3(0)^2 - 3(0) + 0 \\
    y_1 & = 0
    \end{align*}
    \begin{align*}
    y_2 = s(1) & = (x - 1)^3 - 1 \\
    & = (1 - 1)^3 - 1 \\
    y_2 & = -1
    \end{align*}
    \bigskip

    \item\textbf{Problem Statement:} Fine the line of best fit for the following data. What is the error? What does it mean? \\
    \begin{center}
    \begin{tabular}{|c||c|c|}
        \hline
        $i$ & $x_i$ & $y_i$ \\
        \hline
        $1$ & $2$ & $3$ \\
        \hline
        $2$ & $4$ & $7$ \\
        \hline
        $3$ & $6$ & $11$ \\
        \hline
    \end{tabular}
    \end{center}

    \begin{align*}
    \left[\begin{array}{cc}
        2 & 1 \\
        4 & 1 \\
        6 & 1 \\
    \end{array}\right] \times \left[\begin{array}{c}
        a_1 \\
        a_2 \\
    \end{array}\right] & = \left[\begin{array}{c}
        3 \\
        7 \\
        11
    \end{array}\right] \\
    \left[\begin{array}{ccc}
        2 & 4 & 6 \\
        1 & 1 & 1 \\
    \end{array}\right] \times \left[\begin{array}{cc}
        2 & 1 \\
        4 & 1 \\
        6 & 1 \\
    \end{array}\right] \times \left[\begin{array}{c}
        a_1 \\
        a_2 \\
    \end{array}\right] & = \left[\begin{array}{ccc}
        2 & 4 & 6 \\
        1 & 1 & 1 \\
    \end{array}\right] \times \left[\begin{array}{c}
        3 \\
        7 \\
        11
    \end{array}\right] \\
    \left[\begin{array}{cc}
        56 & 12 \\
        12 & 3 \\
    \end{array}\right] \times \left[\begin{array}{cc}
        a_1 \\
        a_2 \\
    \end{array}\right] & = \left[\begin{array}{c}
        100 \\
        21 \\
    \end{array}\right] \\
    \left[\begin{array}{cc|c}
        56 & 12 & 100 \\
        12 & 3 & 21 \\
    \end{array}\right] & \text{swap(R1, R2)} \\
    \left[\begin{array}{cc|c}
        12 & 3 & 21 \\
        56 & 12 & 100 \\
    \end{array}\right] & \text{R2} = \text{R2} - \frac{14}{3}\text{R1}\\
    \left[\begin{array}{cc|c}
        12 & 3 & 21 \\
        0 & -2 & 2 \\
    \end{array}\right]\\
    \end{align*}

    \begin{align*}
        -2a_2 & = 2 \\
        a_2 & = -1
    \end{align*}

    \begin{align*}
        12a_1 + 3a_2 & = 21 \\
        12a_1 - 3 & = 21 \\
        12a_1 & = 24 \\
        a_1 & = 2
    \end{align*}
    The line of best fit is $y = 2x - 1$. The error is $\displaystyle\sum_{i = 1}^3{(y_i - P(x_i))^2} = 0$. This means that the line exactly matches the data points. The error cannot "cancel out" since it is squared, and all of the terms have the same sign.
    \bigskip

    \item\textbf{Problem Statement:} Write a \texttt{Matlab} function, called \texttt{Lagrange\_poly} that inputs a set of data points $(x, y) = (\texttt{datx, daty})$, a set $x$ of numbers at which to interpolate, and outputs the polynomial interpolant, $y$, evaluated at $x$ using Lagrange polynomial interpolation.
    \begin{enumerate}
        \item Use the code to interpolate the following functions:
        \begin{enumerate}
            \item $f_1(x) = e^{-x^2}$
            \item $f_2(x) = \frac{1}{1 + x^2}$
        \end{enumerate}
        using the data points \texttt{datx=-3:1:3}. Interpolate at the points \texttt{x=-3:0.01:3}. Call $P_1$ the Lagrange interpolant of $f_1$ and $P_2$ the Lagrange interpolant of $f_2$. Repeat the experiment except using the data \texttt{datx1=-3:0.5:3}. Call in that case $P_3$ and $P_4$ the new interpolants. Compare the results. \\
        For each interpolation problem, plot on the same graph the function, the two interpolants, and the data set. On a separate plot, plot the error between each interpolant \texttt{y} and \texttt{f(x)}.
        \lstinputlisting[language=Python, caption=Python]{../py/problem6.py}
        \begin{paracol}{2}
            \includegraphics[scale=0.45]{../img/problem6P1P3.png} \\
            \includegraphics[scale=0.45]{../img/problem6P2P4.png}
        \switchcolumn
            \includegraphics[scale=0.45]{../img/problem6P1P3error.png} \\
            \includegraphics[scale=0.45]{../img/problem6P2P4error.png}
        \end{paracol}
        \begin{center}
        \end{center}
    \end{enumerate}
    \bigskip

    \item\textbf{Problem Statement:} Write a \texttt{Matlab} function called \texttt{linear\_spline} which inputs a set of data points $(x, y) = \texttt{(datx, daty)}$, a set $x$ of numbers at which to interpolate, and outputs the polynomial interpolant, $y$, evaluated at $x$ using a linear spline of interpolation.
    \begin{enumerate}
        \item Use the code to interpolate the following functions
        \begin{enumerate}
            \item $f_1(x) = e^{-x^2}$
            \item $f_2(x) = \frac{1}{1 + x^2}$
        \end{enumerate}
        using the data points \texttt{datx=-3:1:3}. Interpolate at the points \texttt{x=-3:0.01:3}. Call $P_5$ the Linear Spline interpolant of $f_1$ and $P_6$ the Linear Spline interpolant of $f_2$. Repeat the experiment except using the data \texttt{datx1=-3:0.5:3}. Call in that case $P_7$ and $P_8$ the new interpolants. Compare the results. \\
        For each interpolation problem, plot on the same graph the function, the two interpolants, and the data set. On a separate plot, plot the error between each interpolant \texttt{y} and \texttt{f(x)}.
        \lstinputlisting[language=Python, caption=Python]{../py/problem7.py}
        \begin{paracol}{2}
            \includegraphics[scale=0.45]{../img/problem7P5P7.png} \\
            \includegraphics[scale=0.45]{../img/problem7P6P8.png}
        \switchcolumn
            \includegraphics[scale=0.45]{../img/problem7P5P7error.png} \\
            \includegraphics[scale=0.45]{../img/problem7P6P8error.png}
        \end{paracol}
    \end{enumerate}
    \bigskip

    \item\textbf{Problem Statement:} Write a \texttt{Matlab} function called \texttt{least\_squares} which inputs a set of data points $(x, y) = \texttt{(datx, daty)}$, the degree of polynomial $n$, and outputs the coefficients $a_i$ as a vector. \\
    Write a second function called \texttt{exp\_least\_squares} which computes the exponential components for least squares as $p_2(x) = a_0e^{a_1x}$ and outputs $a = (a_0, a_1)$ \\
    Write also at the top of the script an anonymous function \texttt{N(d)} which uses a normal distribution. \\
    To create \texttt{daty} in each case, add \texttt{N(0.5)} to each data point.
    \begin{enumerate}
        \item Use the code to approximate the functions
        \begin{enumerate}
            \item $f_1(x) = 2x + 4$ on $[0, 4]$ using $n = 1$.
            \item $f_2(x) = x^2 - 3x + 1$ on $[1, 5]$ using $n = 2$.
        \end{enumerate}
        \item Use the code to approximate
        \begin{enumerate}
            \item $f_3(x) = 3e^{0.25x}$ on $[0, 10]$
        \end{enumerate}
    \end{enumerate}
    \lstinputlisting[language=Python, caption=Python]{../py/problem8.py}
\end{enumerate}

\end{document}