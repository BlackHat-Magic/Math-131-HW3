\documentclass[12pt]{amsart}

% formatting stuff
\addtolength{\hoffset}{-2.25cm}
\addtolength{\textwidth}{4.5cm}
\addtolength{\voffset}{-2.5cm}
\addtolength{\textheight}{5cm}
\setlength{\parskip}{0pt}
\setlength{\parindent}{0pt}

% \usepackage{amsthm}
% \usepackage{amsmath}
\usepackage{amssymb}
\usepackage[colorlinks = true, linkcolor = black, citecolor = black, final]{hyperref}

\usepackage{graphicx}
\usepackage{multicol}
% \usepackage{ marvosym }
% \usepackage{wasysym}
% \usepackage{tikz}
% \usetikzlibrary{patterns}

\usepackage{listings}
\usepackage{xcolor}

\definecolor{codegreen}{rgb}{0,0.6,0}
\definecolor{codegray}{rgb}{0.5,0.5,0.5}
\definecolor{codepurple}{rgb}{0.58,0,0.82}
\definecolor{backcolour}{rgb}{0.95,0.95,0.92}

\lstdefinestyle{mystyle}{
    backgroundcolor=\color{backcolour},   
    commentstyle=\color{codegreen},
    keywordstyle=\color{magenta},
    % numberstyle=\tiny\color{codegray},
    stringstyle=\color{codepurple},
    basicstyle=\ttfamily\footnotesize,
    breakatwhitespace=false,         
    breaklines=true,                 
    captionpos=b,                    
    keepspaces=true,                 
    numbers=none,                    
    numbersep=5pt,                  
    showspaces=false,                
    showstringspaces=false,
    showtabs=false,                  
    tabsize=2
}
\lstset{style=mystyle}

\newcommand{\ds}{\displaystyle}

\setlength{\parindent}{0in}

\pagestyle{empty}

% ----------------------------
% end preamble
% -----------------------------

\begin{document}

{\scshape Math 131} \hfill {\scshape \large Assignment 3} \hfill {\scshape Luke Henderson}
\smallskip
\hrule
\bigskip

For the source code for the coding problems, see the attached Jupyter notebook, Matlab live script. Alternatively, the entire git repository  is attached as a zip archive, and is available \href{https://github.com/blackHat-Magic/math-131-HW3}{on GitHub}. The comments in the code have been omitted here for brevity. They are present in the Jupyter notebook and Matlab live script.

\begin{enumerate}
    \item \textbf{Problem Statement:} Compute the interpolating polynomial $p_2(x)$ that interpolates $f = \sqrt{2}xcos(x)$ at points $x_0 = 0$, $x_1 = \frac{\pi}{4}$, and $x_2 = \frac{\pi}{2}$ in the interval $[0, \frac{\pi}{2}]$.
    \begin{align*}
        p_2(x_0) = 0 & = a(x_0)^2 + b(x_0) + c \\
        0 & = a(0)^2 + b(0) + c \\
        0 & = c \\
        c & = 0 \\
    \end{align*}
    We have $c$, let's move on to calculate $b$.
    \begin{align*}
        p_2(x_1) = \sqrt{2}x_1cos(x_1) & = a(x_1)^2 + b(x_1) + c \\
        \sqrt{2}\frac{\pi}{4}cos\left(\frac{\pi}{4}\right) & = a\left(\frac{\pi}{4}\right)^2 + b\left(\frac{\pi}{4}\right) + 0 \\
        \frac{\pi\sqrt{2}}{4}\frac{\sqrt{2}}{2} & = a\frac{\pi^2}{8} + b\frac{\pi}{4} \\
        \frac{\pi}{4} & = a\frac{\pi^2}{8} + b\frac{\pi}{4} \\
        \frac{\pi}{4} - b\frac{\pi}{4} & = a\frac{\pi^2}{8} \\
        1 - b & = a\frac{\pi^2}{8}\frac{4}{\pi} \\
        1 - b & = a\frac{\pi}{2} \\
        b & = 1 - a\frac{\pi}{2}
    \end{align*}
    We have $b$ in terms of $a$. Using this, we can calculate the value of $a$ and then plug that back in here to get $b$.
    \begin{align*}
        p_2(x_2) = \sqrt{2}x_2cos(x_2) & = a(x_2)^2
    \end{align*}
\end{enumerate}

\end{document}